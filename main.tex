\documentclass{article}
\usepackage{polski}
\usepackage[utf8]{inputenc}
\usepackage{tikz}
\usepackage{listings}
\usepackage[a4paper, total={18cm, 28cm}]{geometry}
\usetikzlibrary{shapes.geometric, arrows}
\usepackage{tcolorbox}

\pagestyle{empty}

\tikzstyle{startstop} = [ellipse, minimum width=1.5cm, minimum height=0.5cm,text centered, draw=black, fill=red!30]
\tikzstyle{trap} = [trapezium, trapezium left angle=70, trapezium right angle=110, minimum width=1.0cm, minimum height=0.5cm, text centered, draw=black, fill=blue!30]
\tikzstyle{pro} = [rectangle, minimum width=1.5cm, minimum height=0.5cm, text centered, draw=black, fill=orange!30]
\tikzstyle{romb} = [diamond, minimum width=1.5cm, minimum height=0.5cm, text centered, draw=black, fill=green!30]
\tikzstyle{arr} = [thick,->,>=stealth]

\newtcolorbox{temp}[2][]{colbacktitle=white,
colback=white,coltitle=black,
title={#2},#1}



\begin{document}

\section*{zadanie 2.1}

\begin{temp}{\large cpp}  
\begin{verbatim}
#define ndk cout<<"\nnaci�nij dowolny klawisz";getch();cout<<"\r                                      \r\n";
#define pln 1
#define eur 4
#define usd 3
#define chf 5
void gotoxy(int x, int y)
{COORD cord;cord.X = x;cord.Y = y;
SetConsoleCursorPosition(GetStdHandle(STD_OUTPUT_HANDLE), cord);}
int main()
{
	setlocale(LC_CTYPE, "Polish");
	cout<<"kwota:";float kwota;cin>>kwota;
	cout<<"\n1.PLN\n2.EUR\n3.USD\n4.CHF";gotoxy(0,1);cout<<"podaj numer twojej walute:";
	char wal1 = tolower(getch());
	cout<<wal1;
	cout<<"\n       \n       \n      \n      ";gotoxy(0,2);		
	int iwal1;
	switch (wal1){
		case '1':
			iwal1 = pln;break;
		case '2':
			iwal1 = eur;break;
		case '3':
			iwal1 = usd;break;
		case '4':
			iwal1 = chf;break;}
	cout<<"\nwarto�� w PLN:"<<kwota*iwal1/pln;
	cout<<"\nwarto�� w EUR:"<<kwota*iwal1/eur;
	cout<<"\nwarto�� w USD:"<<kwota*iwal1/usd;
	cout<<"\nwarto�� w CHF:"<<kwota*iwal1/chf;
	getch();return 0;
}

\end{verbatim}
\end{temp}

\section*{zadanie 2.2}

\begin{temp}{\large cpp}  
\begin{verbatim}
#define ndk cout<<"naciśnij dowolny klawisz";getch();cout<<"\r                                      \r\n";

int main()
{
	setlocale(LC_CTYPE, "Polish");
	float bok = 0;
	float kąt = 0;
	float pole = 0;
	cout<<"Podaj bok: ";cin>>bok;
	cout<<"Podaj liczbę liczbę kątów:";cin>>kąt;
	pole = (bok*bok*kąt)/(4*tan(M_PI/kąt));	
	cout<<"Pole wynosi: "<<pole<<endl;
	ndk;
	return 0;
}

\end{verbatim}
\end{temp}

\section*{zadanie 2.3}

\begin{temp}{\large cpp}  
\begin{verbatim}
#define ndk cout<<"naciśnij dowolny klawisz";getch();cout<<"\r                                      \r\n";

int main()
{
	setlocale(LC_CTYPE, "Polish");
	cout<<"podaj liczbę";int liczba;cin>>liczba;
    int dlugosc=0;
    while(liczba>0){
        liczba/=10;
        dlugosc++;
    }
    switch (dlugosc)
    {
    case 1:
        cout<<"liczba ma 1 cyfrę";
        break;
    case 2:
        cout<<"liczba ma 2 cyfry";
        break;
    case 3:
        cout<<"liczba ma 3 cyfry";
        break;
    case 4:
        cout<<"liczba ma 4 cyfry";
        break;
    case 5:
        cout<<"liczba ma 5 cyfry";
        break;
    case 6:
        cout<<"liczba ma 6 cyfry";
        break;
    case 7:
        cout<<"liczba ma 7 cyfry";
    default:
        cout<<"liczba ma więcej niś 7 cyfr lub mniej";
        break;
    }
    ndk;
	return 0;
}
\end{verbatim}
\end{temp}

\section*{zadanie 2.4}

\begin{temp}{\large cpp}  
\begin{verbatim}
int main() {
    cout<<"a:";int a;cin>>a;
    cout<<"b:";int b;cin>>b;
    cout<<"c:";int c;cin>>c;
    int count=0;
    for(int i=a;i<=b;i++){
        if(i%c==0){
            count++;
			cout<<(i)<<",";} 
    }
    cout<<endl<<count<<" liczb z przedzialu ["<<a<<","<<b<<"] jest podzielne przez "<<c<<endl;
    getchar();
    return 0;
}

\end{verbatim}
\end{temp}

\section*{zadanie 2.4.lekcja}

\begin{temp}{\large cpp}  
\begin{verbatim}
int main() {
	int a;cout<<"A:";cin>>a;
    int b;cout<<"B:";cin>>b;
	int c;cout<<"C:";cin>>c;
	int z = b-a;
	int liczby = 0;
	int i = 0;
	for(i;i<z;i++)
	{
		if ((a+i)%c==0){liczby++;cout<<(a+i)<<",";}
	}
	cout<<endl<<"w zakresie jest "<<liczby<<" liczby podzielnych przez "<<c;
	return 0;
}

\end{verbatim}
\end{temp}

\section*{zadanie 2.5}

\begin{temp}{\large cpp}  
\begin{verbatim}
int main(){
    cout<<"ile liczb chcesz wprowadzic?"<<endl;
    int n;cin>>n;
    float suma = 0;
    for(int i=0;i<n;i++){
        cout<<"podaj liczbe nr "<<i+1<<": ";
        float a;cin>>a; suma += a;}
    cout<<fixed<<setprecision(2)<<"średnia wynosi: "<<suma/n<<endl;
    getchar();
    return 0;
}
\end{verbatim}
\end{temp}

\section*{zadanie 2.6}

\begin{temp}{\large cpp}  
\begin{verbatim}
int main(){
    cout<<"podaj liczbe n:";int n;cin>>n;
    int wynik = 0;
    for(int i=100;i<1000;i++){
        int a=i%10+(i/10)%10+i/100;
        if ((a==n)){
            cout<<i<<endl;
            wynik++;
        }    
    }
    cout<<"ilosc liczb: "<<wynik<<endl;
}
\end{verbatim}
\end{temp}

\section*{zadanie 2.6.time}

\begin{temp}{\large cpp}  
\begin{verbatim}
int main(){
    for(int n=1;n<=27;n++){	
    	int wynik = 0;
	    for(int i=100;i<1000;i++){
	        int a=i%10+(i/10)%10+i/100;
	        if ((a==n)){
	            wynik++;
	        }    
	    }
    cout<<wynik<<endl;
	}
}


\end{verbatim}
\end{temp}

\section*{zadanie 2.7}

\begin{temp}{\large cpp}  
\begin{verbatim}


int main(){
    do{
        cout<<"podaj liczbe n:";int n;cin>>n;
        int wynik;
        for(int i=100;i<1000;i++){
            int a=i%10+(i/10)%10+i/100;
            if ((a==n)){
                cout<<i<<endl;
                wynik++;
            }    
        }
        cout<<"ilosc liczb: "<<wynik<<endl;
        getchar();
    }
    while(true);
}
\end{verbatim}
\end{temp}

\section*{zadanie 2.8}

\begin{temp}{\large cpp}  
\begin{verbatim}
int main(){
	while(true){
		cout<<"ile liczb chcesz wprowadzic?"<<endl;
		bool koniec = false;
		int n;
		while (koniec == false){
			cin>>n;
			if(cin.fail()){cout<<"podana wartosc nie jest liczba"<<endl;}
			else{koniec = true;}
			cin.clear();cin.sync();
		}
		float suma = 0;
		koniec = false;
		int i = 1;
		while(koniec == false){
			cout<<"podaj liczbe nr "<<i<<": ";
			int a;cin>>a;
			if(cin.fail()){cout<<"podana wartosc nie jest liczba"<<endl;}
			else if(a>9 && a<100){
				i++;
				suma += a;
				if(i>n){koniec = true;}
			}
			else{cout<<"podana wartosc nie jest 2 cyfrowa"<<endl;}
			cin.clear();cin.sync();
		}
		cout<<fixed<<setprecision(2)<<"średnia wynosi: "<<suma/n<<endl;
		cout<<"ndk";getch();
	}
	return 0;
}
\end{verbatim}
\end{temp}

\section*{zadanie 2.9}

\begin{temp}{\large cpp}  
\begin{verbatim}

int main() {
    int a,b;
    cout<<"podaj a: ";cin>>a;
    cout<<"podaj b: ";cin>>b;
    //NWW
    int c=a;
    int d=b;
    while (c!=d)
    {
        if (c>d){d=d+b;}
        else if (d>c){c=c+a;}
    }
    cout<<"NWW="<<c<<endl;    
    //NWD
    int nwd = 1;
    int pruba = 1;
    while (pruba <= a && pruba <= b)
    {
        if (a % pruba == 0 && b % pruba == 0){nwd = pruba;}
        pruba++;
    }
    if (nwd>1){cout<<"NWD="<<nwd<<endl;}
    else {cout<<"NWD="<<"brak"<<endl;}         
}
\end{verbatim}
\end{temp}
\end{document}
